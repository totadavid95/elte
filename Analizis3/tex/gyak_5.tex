\documentclass[a4paper]{article}
\usepackage[textwidth=170mm, textheight=230mm, margin=20mm, bottom=25mm]{geometry}
\usepackage[normalem]{ulem}
\usepackage[utf8]{inputenc}
\usepackage[T1]{fontenc}
\usepackage[magyar]{babel}
\usepackage{amsmath, amsthm, amssymb, hyperref}
\usepackage[hu]{datetime}
\usepackage{enumitem}
\usepackage{lmodern}
\usepackage{xparse}
\usepackage{multicol}
\usepackage{relsize}
\usepackage{framed}

\DeclareSymbolFont{lettersA}{U}{txmia}{m}{it}
\makeatletter
\DeclareMathSymbol{\m@thbbch@rC}{\mathord}{lettersA}{131}
\DeclareMathSymbol{\m@thbbch@rN}{\mathord}{lettersA}{142}
\DeclareMathSymbol{\m@thbbch@rQ}{\mathord}{lettersA}{145}
\DeclareMathSymbol{\m@thbbch@rR}{\mathord}{lettersA}{146}
\DeclareMathSymbol{\m@thbbch@rZ}{\mathord}{lettersA}{154}
\makeatother
\ExplSyntaxOn
\NewDocumentCommand{\varmathbb}{m}
{
	\tl_map_inline:nn { #1 } { \use:c { m@thbbch@r##1 } }
}
\ExplSyntaxOff

\input{glyphtounicode}
\pdfgentounicode=1
\hypersetup{colorlinks = true}
\setlength\parindent{0pt}
\setlength{\parskip}{1em}
\def\Z{\mathbb{Z}}
\def\Q{\mathbb{Q}}
\def\R{\mathbb{R}}
\def\C{\mathbb{C}}
\def\N{\mathbb{N}}

\def\ZZ{\varmathbb{Z}}
\def\QQ{\varmathbb{Q}}
\def\RR{\varmathbb{R}}
\def\CC{\varmathbb{C}}
\def\NN{\varmathbb{N}}

\newtheoremstyle{qstyle}{1.5em}{-1em}{\bfseries\boldmath}{}{\unboldmath\bfseries}{.}{ }{}
\theoremstyle{qstyle}
\newtheorem{question}{}{}
\setlist[itemize]{topsep=0pt}

\begin{document}
	\begin{center}
		{\Large\textbf{Analízis 3. (B és C szakirány)}}\\
		{\Large Szükséges ismeretek az 5. gyakorlathoz}
	\end{center}
	
	\begin{framed}
		\textbf{Jelen dokumentum ekkor lett frissítve: {\yyyymmdddate\today} \ \currenttime}\\
		További kidolgozások elérhetőek \href{https://people.inf.elte.hu/totadavid95/Analizis3/2019-tavasz/}{ide kattintva}. A gyakorlatok anyaga \href{http://numanal.inf.elte.hu/~szili/Oktatas/An3_BC_szakirany_2019/An3_BC_gyak_2019_tavasz.pdf}{ide kattintva} érhető el.\\	
		Forrás(ok): \href{http://numanal.inf.elte.hu/~szili/Oktatas/An3_BC_szakirany_2019/An3_BC_ea_def_tetel_2019_tavasz.pdf}{Dr. Szili László - Definíciók és tételek az előadásokon}
	\end{framed}

	\begin{question}
		Definiálja a metrikus teret.  
	\end{question}
	Az $(M,\varrho)$ rendezett pár metrikus tér, ha $M\ne \emptyset$ halmaz, és
	$$\varrho: M \times M \to \R$$
	olyan függvény, melyre $\forall x,y,z\in M$ esetén
	\vspace{-4mm}
	\begin{enumerate}
		\item $\varrho(x,y)\ge 0$,
		\item $\varrho(x,y) = 0 \Longleftrightarrow x = y$,
		\item $\varrho(x,y) = \varrho(y,x)$ (szimmetria),
		\item $\varrho(x,y) \le \varrho(x,z) + \varrho(z,y)$ (háromszög-egyenlőtlenség).
	\end{enumerate}
	\vspace{-4mm}
	A $\varrho(x,y)$ az $x,y$ pontok távolsága, a $\varrho$ pedig a távolságfüggvény (vagy metrika).
	
	\begin{question}
		Hogyan értelmezzük $\R^n$-ben a $\varrho_2$ euklideszi metrikát?
	\end{question}
	Ha $1\le n \in \N$ és $x (x_1,\dots,x_n)$,$y=(y_1,\dots,y_n)\in \R^n$, akkor
	$$\varrho_2(x,y) := \sqrt{\sum_{k=1}^{n}|x_k -y_k|^2} \text{.}$$


	\begin{question}
		Írja le a normált tér definícióját.
	\end{question}
	Az $(X,||.||)$ rendezett pár normált tér, ha
	\vspace{-4mm}
	\begin{enumerate}
		\item $X \ne \emptyset$ lineáris tér (v. vektortér) az $\R$ számtest felett;
		\item $||.|| : X\to  \R$ olyan függvény, amelyre $\forall x,y \in X$ és $\forall \lambda \in \R$ esetén
		\begin{itemize}
			\item $||x||\ge 0$,
			\item $||x|| = 0 \Leftrightarrow x = \boldsymbol{0} \quad$ ($\boldsymbol{0}$ az $X$ lin. tér nulleleme)
			\item $||\lambda x|| = |\lambda | \cdot ||x||$,
			\item $||x+y|| \le ||x||+||y||$ (háromszög-egyenlőtlenség)
		\end{itemize}
	\end{enumerate}
	\vspace{-4mm}
	A $||.||$ leképezést normának, az $||x||$ számot pedig az $x$ elem normájának mondjuk.
	
	\begin{question}
		Definiálja $\R^n$-en a $||.||_p$ normákat.
	\end{question}
	Ha $1\le n\in\N, 1\le p\le +\infty$ és $x = (x_1,\dots,x_n) \in \R^n$, akkor
	
	$$
	||x||_p := 
	\left\{\begin{array}{rl}
	\left(\mathlarger{\sum_{k=1}^{n}|x_k|^p} \right)^{\frac{1}{p}} & \text{ ha } 1\le p < +\infty \\
	&\\
	\max\limits_{1 \le k\le n} |x_k| & \text{ ha } p = +\infty\text{.} 
	\end{array}\right.
	$$
	
	\begin{question}
		Definiálja normált térben a konvergens sorozat fogalmát.
	\end{question}
	Az $(X,||.||)$ normált tér egy $(a_n):\N\to X$ sorozata konvergens, ha
	$$\exists \alpha\in X \text{, hogy }\quad \forall \varepsilon > 0 \text{ számhoz } \quad \exists n_0 \in \N \text{, hogy } \quad \forall n\ge n_0 \text{ indexre } ||a_n-\alpha|| < \varepsilon \text{.}$$
	
	Ekkor $\alpha$ az $(a_n)$ határértéke.
	
	
	\begin{question}
		Mit jelent az, hogy két norma ekvivalens?
	\end{question}
	Az $X$ lineáris téren adott $||.||^{(1)}$ és $||.||^{(2)}$ normák ekvivalensek, ha léteznek olyan $m,M$ pozitív valós számok, hogy
	$$m\cdot ||x||^{(1)}\le ||x||^{(2)}\le M\cdot ||x||^{(1)}\quad\quad (x\in X)\text{.}$$
	Jelölés: $||.||^{(1)} \sim ||.||^{(2)}$.

	\begin{question}
		Milyen állítást ismer ekvivalens normák esetén sorozatok konvergenciájára?
	\end{question}
	Tegyük fel, hogy az $X$ lineáris téren értelmezett $||.||^{(1)}$ és $||.||^{(2)}$ normák ekvivalensek. Ekkor tetszőleges $(a_n):\N\to X$ sorozatra
	$$\lim(a_n) \overset{\hbox{$||.||^{(1)}$}}{=} \alpha \Longleftrightarrow \lim(a_n) \stackrel{\hbox{$||.||^{(2)}$}}{\hbox{=}} \alpha \text{.} $$

\end{document}