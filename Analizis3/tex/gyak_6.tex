\documentclass[a4paper]{article}
\usepackage[textwidth=170mm, textheight=230mm, margin=20mm, bottom=25mm]{geometry}
\usepackage[normalem]{ulem}
\usepackage[utf8]{inputenc}
\usepackage[T1]{fontenc}
\usepackage[magyar]{babel}
\usepackage{amsmath, amsthm, amssymb, hyperref}
\usepackage[hu]{datetime}
\usepackage{enumitem}
\usepackage{lmodern}
\usepackage{xparse}
\usepackage{multicol}
\usepackage{framed}

\DeclareSymbolFont{lettersA}{U}{txmia}{m}{it}
\makeatletter
\DeclareMathSymbol{\m@thbbch@rC}{\mathord}{lettersA}{131}
\DeclareMathSymbol{\m@thbbch@rN}{\mathord}{lettersA}{142}
\DeclareMathSymbol{\m@thbbch@rQ}{\mathord}{lettersA}{145}
\DeclareMathSymbol{\m@thbbch@rR}{\mathord}{lettersA}{146}
\DeclareMathSymbol{\m@thbbch@rZ}{\mathord}{lettersA}{154}
\makeatother
\ExplSyntaxOn
\NewDocumentCommand{\varmathbb}{m}
{
	\tl_map_inline:nn { #1 } { \use:c { m@thbbch@r##1 } }
}
\ExplSyntaxOff

\input{glyphtounicode}
\pdfgentounicode=1
\hypersetup{colorlinks = true}
\setlength\parindent{0pt}
\setlength{\parskip}{1em}
\def\Z{\mathbb{Z}}
\def\Q{\mathbb{Q}}
\def\R{\mathbb{R}}
\def\C{\mathbb{C}}
\def\N{\mathbb{N}}

\def\ZZ{\varmathbb{Z}}
\def\QQ{\varmathbb{Q}}
\def\RR{\varmathbb{R}}
\def\CC{\varmathbb{C}}
\def\NN{\varmathbb{N}}

\newtheoremstyle{qstyle}{1.5em}{-1em}{\bfseries\boldmath}{}{\unboldmath\bfseries}{.}{ }{}
\theoremstyle{qstyle}
\newtheorem{question}{}{}
\setlist[itemize]{topsep=0pt}

\begin{document}
	\begin{center}
		{\Large\textbf{Analízis 3. (B és C szakirány)}}\\
		{\Large Szükséges ismeretek a 6. gyakorlathoz}
	\end{center}
	
	\begin{framed}
		\textbf{Jelen dokumentum ekkor lett frissítve: {\yyyymmdddate\today} \ \currenttime}\\
		További kidolgozások elérhetőek \href{https://people.inf.elte.hu/totadavid95/Analizis3/2019-tavasz/}{ide kattintva}. A gyakorlatok anyaga \href{http://numanal.inf.elte.hu/~szili/Oktatas/An3_BC_szakirany_2019/An3_BC_gyak_2019_tavasz.pdf}{ide kattintva} érhető el.\\	
		Forrás(ok): \href{http://numanal.inf.elte.hu/~szili/Oktatas/An3_BC_szakirany_2019/An3_BC_ea_def_tetel_2019_tavasz.pdf}{Dr. Szili László - Definíciók és tételek az előadásokon}
	\end{framed}
	
	\begin{question}
		Definiálja a normált terek közötti leképezések pontbeli folytonosságát.
	\end{question}
	Legyen $(X,||.||_X)$ és $(Y,||.||_Y)$ normált tér. Az $f\in X\to Y$ függvény folytonos az $a\in \mathcal{D}_f$ pontban (jelölés: $f\in C\{a\}$), ha
	$$\forall \varepsilon > 0 \text{ számhoz } \exists \delta > 0, \quad \forall x \in K_{\delta}^{||.||_X}(a) \cap \mathcal{D}_f \text{ esetén } f(x)\in K_{\varepsilon}^{||.||_Y}(f(a)) \text{,}$$
	azaz, ha
	$$\forall \varepsilon > 0 \text{ számhoz } \exists \delta > 0, \quad \forall x \in \mathcal{D}_f, \quad ||x-a||_X < \delta \text{ esetén } ||f(x)-f(a)||_Y < \varepsilon \text{.}$$


	\begin{question}
		Mit jelent egy $\R^2\to\R^1$ függvény pontbeli folytonossága?
	\end{question}
	Az $f\in\R^2\to\R$ függvény folytonos az $a\in\mathcal{D}_f$ pontban, ha
	
	$$\forall \varepsilon > 0 \text{ számhoz } \exists \delta > 0, \quad \forall x \in \mathcal{D}_f, \quad ||x-a|| < \delta \text{ esetén } |f(x)-f(a)| < \varepsilon \text{,}$$
	ahol $||.||$ tetszőleges norma az $\R^2$ lineáris téren.
	
	\begin{question}
		Hogyan szól a folytonosságra vonatkozó átviteli elv?
	\end{question}
	Legyen $(X,||.||_X)$ és $(Y,||.||_Y)$ normált tér, $f\in X\to Y$ és $a\in\mathcal{D}_f$. Ekkor:
	\vspace{-4mm}
	\begin{enumerate}
		\item $f\in C\{a\} \Longleftrightarrow \forall (x_n): \N\to\mathcal{D}_f,\quad \lim\limits_{n\to+\infty}x_n \overset{\hbox{$||.||_X$}}{=}  a \text{ sorozatra } \lim\limits_{n\to+\infty}f(x_n) \overset{\hbox{$||.||_Y$}}{=} f(a)$.
		\item Tegyük fel, hogy $(x_n):\N\to\mathcal{D}_f, \lim\limits_{n\to+\infty}x_n \overset{\hbox{$||.||_X$}}{=}  a \text{ és } \lim\limits_{n\to+\infty}f(x_n) \overset{\hbox{$||.||_Y$}}{\ne} f(a)$. Ekkor az $f$ függvény nem folytonos $a$-ban.
	\end{enumerate}
	\vspace{-4mm}

	\begin{question}
		Írja le a torlódási pont definícióját.
	\end{question}
	Legyen $(X,||.||)$ normált tér és $\emptyset \ne A\subset X$. Ekkor $a\in X$ torlódási pontja az $A$ halmaznak (jelölés: $a\in A'$), ha
	$$\forall K(a)\subset X \text{ környezetre } K(a)\cap A \text{ végtelen halmaz.}$$

	\begin{question}
		Írja le normált terek közötti leképezésekre a határérték definícióját.
	\end{question}
	Legyen $(X,||.||_X)$ és $(Y,||.||_Y)$ normált tér. Az $f \in X\to Y$ függvénynek az $a\in\mathcal{D}'_f$ pontban van határértéke, ha létezik olyan $A\in Y$, hogy az
	$$
	\widetilde{f}(x) = \left\{\begin{array}{rl}
	f(x), & \text{ha } x\in\mathcal{D}_f\backslash\{a\}\\
	A, & \text{ha } x=a
	\end{array}\right.
	$$
	függvény folytonos az $a\in\mathcal{D}_{\widetilde{f}}$ pontban. Ha létezik ilyen $A$, akkor az egyértelmű, és azt az $f$ függvény $a$-beli határértékének nevezzük. (Jelölés: $\lim\limits_{a}f = A$).

	\begin{question}
		Fogalmazza meg a függvények határértékére vonatkozó átviteli elvet.
	\end{question}
	Legyen $(X,||.||_X)$ és $(Y,||.||_Y)$ normált tér, $f\in X\to Y$ és $a\in\mathcal{D}'_f$. Ekkor:
	\vspace{-4mm}
	\begin{enumerate}
		\item $\lim\limits_{a}f = A \Longleftrightarrow \forall (x_n):\N\to\mathcal{D}_f\backslash\{a\}, x_n \xrightarrow[n \to +\infty]{||.||_X} a$ esetén $f(x_n) \xrightarrow[n \to +\infty]{||.||_Y} A$.
		\item Tegyük fel, hogy a $\mathcal{D}_f\backslash\{a\}$ halmazbeli $(x_n)$ és $(u_n)$ sorozatok mindegyike az $a\in\mathcal{D}'_f$ ponthoz konvergál és
		$$\lim\limits_{x\to+\infty}f(x_n) \ne \lim\limits_{n\to+\infty}f(u_n)\text{.}$$
		Ekkor az $f$ függvénynek nincs határértéke $a$-ban.
	\end{enumerate}
	\vspace{-4mm}

	
\end{document}