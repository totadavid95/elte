\documentclass[a4paper]{article}
\usepackage[textwidth=170mm, textheight=230mm, margin=20mm, bottom=25mm]{geometry}
\usepackage[normalem]{ulem}
\usepackage[utf8]{inputenc}
\usepackage[T1]{fontenc}
\usepackage[magyar]{babel}
\usepackage{amsmath, amsthm, amssymb, hyperref}
\usepackage[hu]{datetime}
\usepackage{enumitem}
\usepackage{lmodern}
\usepackage{xparse}
\usepackage{multicol}
\usepackage{framed}

\DeclareSymbolFont{lettersA}{U}{txmia}{m}{it}
\makeatletter
\DeclareMathSymbol{\m@thbbch@rC}{\mathord}{lettersA}{131}
\DeclareMathSymbol{\m@thbbch@rN}{\mathord}{lettersA}{142}
\DeclareMathSymbol{\m@thbbch@rQ}{\mathord}{lettersA}{145}
\DeclareMathSymbol{\m@thbbch@rR}{\mathord}{lettersA}{146}
\DeclareMathSymbol{\m@thbbch@rZ}{\mathord}{lettersA}{154}
\makeatother
\ExplSyntaxOn
\NewDocumentCommand{\varmathbb}{m}
{
	\tl_map_inline:nn { #1 } { \use:c { m@thbbch@r##1 } }
}
\ExplSyntaxOff

\input{glyphtounicode}
\pdfgentounicode=1
\hypersetup{colorlinks = true}
\setlength\parindent{0pt}
\setlength{\parskip}{1em}
\def\Z{\mathbb{Z}}
\def\Q{\mathbb{Q}}
\def\R{\mathbb{R}}
\def\C{\mathbb{C}}
\def\N{\mathbb{N}}

\def\ZZ{\varmathbb{Z}}
\def\QQ{\varmathbb{Q}}
\def\RR{\varmathbb{R}}
\def\CC{\varmathbb{C}}
\def\NN{\varmathbb{N}}

\newtheoremstyle{qstyle}{1.5em}{-1em}{\bfseries\boldmath}{}{\unboldmath\bfseries}{.}{ }{}
\theoremstyle{qstyle}
\newtheorem{question}{}{}
\setlist[itemize]{topsep=0pt}

\begin{document}
	\begin{center}
		{\Large\textbf{Analízis 3. (B és C szakirány)}}\\
		{\Large Szükséges ismeretek a 4. gyakorlathoz}
	\end{center}
	
	\begin{framed}
		\textbf{Jelen dokumentum ekkor lett frissítve: {\yyyymmdddate\today} \ \currenttime}\\
		További kidolgozások elérhetőek \href{https://people.inf.elte.hu/totadavid95/Analizis3/2019-tavasz/}{ide kattintva}. A gyakorlatok anyaga \href{http://numanal.inf.elte.hu/~szili/Oktatas/An3_BC_szakirany_2019/An3_BC_gyak_2019_tavasz.pdf}{ide kattintva} érhető el.\\	
		Forrás(ok): \href{http://numanal.inf.elte.hu/~szili/Oktatas/An3_BC_szakirany_2019/An3_BC_gyak_2019_tavasz.pdf}{Dr. Szili László - Analízis 3. gyakorlatok}, \href{https://people.inf.elte.hu/totadavid95/Analizis3/Anal3_def.pdf}{2018 őszi kidolgozás}
	\end{framed}

	\begin{question}
		Adjon meg olyan függvényt, amelynek \emph{nincs} primitív függvénye.
	\end{question}
	$f(x)=\mathrm{sign}(x)$   \quad  $(x \in \left(-1,1)\right)$
	
	\begin{question}
		Adjon meg egy példát \emph{nem integrálható} függvényre. 
	\end{question}
	Legyen
	$$
	f(x) := 
	\begin{cases}
	1, \text{ ha } x \in \Q \\
	0, \text{ ha } x \in \R \setminus \Q \text{.} \\
	\end{cases}
	$$
	Ekkor $f \notin R[0,1]$.
	
	\begin{question}
		Hogyan szól a Riemann-integrálható függvények hányadosával kapcsolatban tanult tétel?
	\end{question}
	Legyen $f, g \in R[a,b]$ tetszőleges és tegyük fel, hogy valamilyen $m > 0$ számmal $|g(x)| \ge m \quad (x \in [a,b])$. Ekkor $\frac{f}{g} \in R[a,b]$.
	
	\begin{question}
		Értelmezze síkidom területét.
	\end{question}
	Ha a korlátos $f : [a,b] \to \R$ függvény Riemann-integrálható az $[a,b]$ intervallumon és $f(x)\ge 0 \quad (x\in [a,b])$, akkor az $f$ grafikonja alatti
	$$A:= \{(x.y)\in\R^2 | a \le x \le b, 0\le y \le f(x)\}$$
	síkidom területét így \textit{értelmezzük}:
	$$t(A) := \int_{a}^{b} f(x)dx \text{.}$$

	\begin{question}
		Hogyan számítja ki forgástest térfogatát?
	\end{question}
	Legyen $f : [a,b] \to \R$ folytonos függvény és tegyük fel, hogy $f \ge 0$  az $[a,b]$ intervallumon. Az $f$ grafikonjának az $x$-tengely körüli forgatásával adódó
	$$H:= \{(x,y,z)\in\R^3 | a\le x\le b, y^2+z^2 \le f^2(x)  \}$$
	forgástest térfogata:
	$$V(H):= \pi \int_{a}^{b} f^2(x)dx\text{.}$$
	
	\begin{question}
		Mit tud mondani függvénygrafikon hosszának a kiszámításáról?
	\end{question}
	Legyen $a,b \in \R, a<b$. Ha az $f:[a,b] \to \R$ függvény folytonosan deriválható az $[a,b]$ intervallumon, akkor $f$ grafikonjának van ívhossza és az egyenlő az
	$$\int_{a}^{b}\sqrt{1+[f'(x)]^2} dx$$
	határozott integrállal.
	
\end{document}