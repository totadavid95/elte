\documentclass[a4paper]{article}
\usepackage[textwidth=170mm, textheight=230mm, margin=20mm, bottom=25mm]{geometry}
\usepackage[normalem]{ulem}
\usepackage[utf8]{inputenc}
\usepackage[T1]{fontenc}
\usepackage[magyar]{babel}
\usepackage{amsmath, amsthm, amssymb, hyperref}
\usepackage[hu]{datetime}
\usepackage{enumitem}
\usepackage{lmodern}
\usepackage{xparse}
\usepackage{multicol}
\usepackage{relsize}
\usepackage{framed}
\usepackage{centernot}
\usepackage{mathtools}
\usepackage{commath}
%\usepackage[sc,osf]{mathpazo}

\DeclareSymbolFont{lettersA}{U}{txmia}{m}{it}
\makeatletter
\DeclareMathSymbol{\m@thbbch@rC}{\mathord}{lettersA}{131}
\DeclareMathSymbol{\m@thbbch@rN}{\mathord}{lettersA}{142}
\DeclareMathSymbol{\m@thbbch@rQ}{\mathord}{lettersA}{145}
\DeclareMathSymbol{\m@thbbch@rR}{\mathord}{lettersA}{146}
\DeclareMathSymbol{\m@thbbch@rZ}{\mathord}{lettersA}{154}
\makeatother
\ExplSyntaxOn
\NewDocumentCommand{\varmathbb}{m}
{
	\tl_map_inline:nn { #1 } { \use:c { m@thbbch@r##1 } }
}
\ExplSyntaxOff

\input{glyphtounicode}
\pdfgentounicode=1
\hypersetup{colorlinks = true}
\setlength\parindent{0pt}
\setlength{\parskip}{1em}
\def\Z{\mathbb{Z}}
\def\Q{\mathbb{Q}}
\def\R{\mathbb{R}}
\def\C{\mathbb{C}}
\def\N{\mathbb{N}}

\def\ZZ{\varmathbb{Z}}
\def\QQ{\varmathbb{Q}}
\def\RR{\varmathbb{R}}
\def\CC{\varmathbb{C}}
\def\NN{\varmathbb{N}}

\newtheoremstyle{qstyle}{1.5em}{-1em}{\bfseries\boldmath}{}{\unboldmath\bfseries}{.}{ }{}
\theoremstyle{qstyle}
\newtheorem{question}{}{}
\setlist[itemize]{topsep=0pt}

% normák
\let\oldnorm\norm   % <-- Store original \norm as \oldnorm
\let\norm\undefined % <-- "Undefine" \norm
\DeclarePairedDelimiter\norm{\lVert}{\rVert}

\begin{document}
	\begin{center}
		{\Large\textbf{Analízis 3. (B és C szakirány)}}\\
		{\Large Szükséges ismeretek a 9. gyakorlathoz}
	\end{center}
	
	\begin{framed}
		\textbf{Jelen dokumentum ekkor lett frissítve: {\yyyymmdddate\today} \ \currenttime}\\
		További kidolgozások elérhetőek \href{https://people.inf.elte.hu/totadavid95/Analizis3/2019-tavasz/}{ide kattintva}. A gyakorlatok anyaga \href{http://numanal.inf.elte.hu/~szili/Oktatas/An3_BC_szakirany_2019/An3_BC_gyak_2019_tavasz.pdf}{ide kattintva} érhető el.\\	
		Forrás(ok): \href{http://numanal.inf.elte.hu/~szili/Oktatas/An3_BC_szakirany_2019/An3_BC_ea_def_tetel_2019_tavasz.pdf}{Dr. Szili László - Definíciók és tételek az előadásokon}
	\end{framed}

	\begin{question}
		Fogalmazza meg a Lagrange-féle középértéktételt.
	\end{question}
	Legyen $f \in \R^n \to \R (n \in \N^+)$ és $a \in int\mathcal{D}_f$. Tegyük fel, hogy $\exists K(a) \subset \mathcal{D}_f$, hogy $f \in D(K(a))$. Legyen $h \in \R^n$ olyan vektor, amelyre $a+h \in K(a)$. Ekkor
	$$\exists \nu \in (0,1) \text{ úgy, hogy } f(a+h)-f(a)=f'(a+\nu h) \cdot h = \langle{f'(a+\nu h),h}\rangle \text{.}$$
	
	\begin{question}
		Mit jelent az, hogy egy függvény kétszer deriválható egy pontban?
	\end{question}
	Az $f \in \R^n \to \R \quad (2 \le n \in \N)$ függvény kétszer deriválható az $a \in int\mathcal{D}_f$ pontban (jelben: $f \in D^2\{a\})$, ha
	\vspace{-4mm}
	\begin{enumerate}
		\item $\exists K(a) \subset \mathcal{D}_f$, hogy $f \in D\{x\}$ minden $x \in K(a)$ pontban, és
		\item $\forall i = 1,2,\dots,n$ indexre $\partial_i f \in D\{a\}$.
	\end{enumerate}
	\vspace{-4mm}
	
	\begin{question}
		Definiálja a Hesse-féle mátrixot.
	\end{question}
	Ha az $f \in \R^n \to \R \quad (2 \le n \in \N)$ függvény kétszer deriválható az $a \in int\mathcal{D}_f$ pontban, akkor az
	
	$$f''(a) = \begin{bmatrix} 
	\partial_{11}f(a) & \partial_{12}f(a) & \dots & \partial_{1n}f(a) \\
	\partial_{21}f(a) & \partial_{22}f(a) & \dots & \partial_{2n}f(a) \\
	\vdots & \vdots &  \vdots &  \vdots  \\
	\partial_{n1}f(a) & \partial_{n2}f(a) & \dots & \partial_{nn}f(a) \\
	\end{bmatrix} \in \R^{n \times n}$$

	szimmetrikus mátrixot az $f$ függvény $a$ pontbeli Hesse-féle mátrixának nevezzük.
	
	\begin{question}
		Fogalmazza meg a Young-tételt.
	\end{question}
	Ha $f \in \R^n \to \R \quad (2 \le n \in \N)$ és $f \in D^2\{a\}$, akkor
	$$\partial_{ij}f(a) = \partial_{ji}f(a) \quad \forall i,j = 1,\dots,n \text{ indexre.}$$

	\begin{question}
		Mit jelent az, hogy egy függvény s-szer deriválható egy pontban?
	\end{question}
	Az $f \in \R^n \to \R \quad (2 \le n \in \N)$ függvény $s$-szer $(2 \le s \in \N)$ deriválható az $a \in int\mathcal{D}_f$ pontban (jelben: $f \in D^s\{a\})$, ha
	\vspace{-4mm}
	\begin{enumerate}
		\item $\exists K(a) \subset \mathcal{D}_f$, hogy $f \in D^{s-1}(K(a))$ és
		\item minden $(s-1)$-edrendű
			$$\partial_{i_1} \partial_{i_2} \dots \partial_{i_{s-1}}f \quad (1 \le i_1,i_2,\dots,i_{s-1} \le n)$$
		parciális deriváltfüggvény deriválható az $a$ pontban.
	\end{enumerate}
	\vspace{-4mm}
	
	
	\begin{question}
		Adja meg a Taylor-polinom definícióját.
	\end{question}
	Legyen $f \in \R^n \to \R \quad (n \in N^+)$ és $a \in int\mathcal{D}_f$. Tegyük fel, hogy egy $m \in \N$ számra $f \in D^m\{a\}$. Az $f$ függvény $a$ ponthoz tartozó $m$-edfokú, $n$-változós Taylor-polinomját így értelmezzük:
	$$(T_{a,m}f)(a+h) := f(a) + \sum_{k=1}^{m}\left(\sum_{|i| = k} \frac{\partial^i f(a)}{i!}h^i\right) \quad (h \in \R^n) \text{.}$$
	Ha $m=0$, akkor $T_{a,0}f \equiv f(a)$, továbbá $(T_{a,m}f)(a)=f(a)$.
	
	\newpage
	
	\begin{question}
		Fogalmazza meg a Taylor-formulát a Lagrange-féle maradéktaggal.
	\end{question}
	Legyen $f \in \R^n \to \R \quad (n \in \N^+)$ és $a \in int\mathcal{D}_f$. Tegyük fel, hogy egy $m \in \N$ számmal $f \in D^{m+1}(K(a))$ teljesül. Ekkor $\forall h \in \R^n \quad (a+h \in K(a))$ vektorhoz $\exists \nu \in (0,1)$, amelyre
	$$f(a+h)=(T_{a,m}f)(a+h)+\sum_{|i|=m+1}\frac{\partial^i f(a + \nu h)}{i!}h^i \text{.}$$
	
	\begin{question}
		Fogalmazza meg a Taylor-formulát Peano-féle maradéktaggal.
	\end{question}
	Tegyük fel, hogy az $f \in \R^n \to \R \quad (n \in \N^+)$ függvényre az $a \in int\mathcal{D}_f$ pontban egy $m \in \N^+$ számmal $f \in D^m\{a\}$ teljesül. Ekkor: $\exists \varepsilon \in \R^n \to \R$, a $\lim\limits_{0} \varepsilon = 0$ feltételnek eleget tevő függvény, amelyre
	$$f(a+h)=(T_{a,m}f)(a+h)+\varepsilon (h) \cdot \norm{h}^m \quad (h \in \R^n, a+h \in \mathcal{D}_f)$$
	ahol $\norm{.}$ tetszőleges norma $\R^n$-en.
	
	\vspace{5mm}
	\begin{framed}
		\begin{center}
			\Large Feladatmegoldások (Csörgő István gyakorlata)
		\end{center}
	\end{framed}
	
	\textbf{1.} A definíció alapján lássa be, hogy az $f$ függvény totálisan deriválható az $a \in int\mathcal{D}_f$ pontban, és adja meg az $f'(a)$ deriváltmátrixot. Az $f'(a)$-ra így kapott eredményt ellenőrizze a Jacobi-mátrix kiszámításával.
	
	\textbf{a)} $f(x,y) = 2x^2+3xy-y^2 \quad\quad\quad a = (1,2)$
	
	\begin{framed}
		$$f \in D(a) \Longleftrightarrow \exists A \in \R^{m \times n} \quad \lim\limits_{h \to 0} \frac{f(a+h)-f(a)-A\cdot h}{\norm{h}} = 0$$
		$$\varepsilon(h) = f(a+h)-f(a)-A\cdot h \quad \longrightarrow \quad \varepsilon(h) + A\cdot h = f(a+h)-f(a)$$
	\end{framed}

	$f(1,2) = 2\cdot 1^2+3\cdot 1 \cdot 2-2^2 = 2 + 6 - 4 = 4$
	
	$f(a+h)-f(a) = f((1,2)+(h_1,h_2)) - f(1,2) = f(1+h_1, 2+h_2)-4 = 2(1+h_1)^2 + 3(1+h_1)(2+h_2)-(2+h_2)^2-4=2(1+2h_1+h_1^2)+3(2+h_2+2h_1+h_1h_2)-(4+4h_2+h_2^2)-4=2+4h_1+2h_1^2+6+3h_2+6h_1+3h_1h_2-4-4h_2-h_2^2-4=10h_1-h_2+2h_1^2+3h_1h_2-h_2^2= \begin{bmatrix}10 & -1\end{bmatrix} \begin{bmatrix}h_1 \\ h_2 \end{bmatrix} + \underbrace{2h_1^2+3h_1h_2-h_2^2}_{\varepsilon(h)}$
	
	$$\left|\frac{\varepsilon(h)}{\norm{h}}-0\right| = \frac{|\varepsilon(h)|}{\norm{h}} = \frac{|2h_1^2+3h_1h_2-h_2^2|}{\norm{h}} \le \frac{2\norm{h}^2 + 3\norm{h} \cdot \norm{h} + \norm{h}^2}{\norm{h}} = 6 \cdot \norm{h} = 6 \cdot \norm{h-(0,0)} \xrightarrow[]{h \to 0} 0 $$
	
	Tehát $f \in D(a)$ és $f'(a) = \begin{bmatrix}10 & -1\end{bmatrix} \in \R^{1 \times 2}$.
	
	\underline{\textit{Ellenőrzés:}}
	$$\partial_1f=4x+3y \quad \longrightarrow \quad \partial_1f(1,2)=4+1+3\cdot 2 = 10 $$
	$$\partial_2f=3x-2y \quad \longrightarrow \quad \partial_2f(1,2)=3 \cdot 1 - 2 \cdot 2 = -1$$
	
	\newpage
	
	\textbf{3.} Legyen
	$$
	f(x,y) := \left\{\begin{array}{rl}
	\frac{x \cdot y^2}{x^2 + y^2}, & \text{ha } (x,y) \in \R^2 \backslash \{(0,0)\}\\
	0, & \text{ha } (x,y) = (0,0) \text{.}
	\end{array}\right.
	$$
	Mutassa meg, hogy az $f$ függvény a $(0,0)$ pontban ...
	
	\textbf{a)} ... folytonos
	
	\underline{\textit{Áll.:}} $f \in C(0,0)$
	
	\underline{\textit{Biz.:}} $\lim\limits_{(x,y)\to (0,0)} \frac{xy^2}{x^2+y^2} = 0$, mivel
	$$\left|\frac{xy^2}{x^2+y^2}-0\right| = \frac{|x|\cdot |y|^2}{\norm{(x,y)}^2} \le \frac{\norm{(x,y)} \cdot \norm{(x,y)}^2}{\norm{(x,y)}^2} = \norm{(x,y)-(0,0)} \xrightarrow[]{(x,y) \to (0,0)} 0$$
	
	\textbf{c)} ... totálisan nem deriválható
	
	\underline{\textit{Áll.:}} $f \notin D(0,0)$
	
	\underline{\textit{Biz.:}}
	$$\partial_1f(0,0) = \left(\frac{d}{dx}\underbrace{f(x,0)}_{0}\right)_{x=0} = \left(\frac{d}{dx}0\right)_{x=0} = 0 \quad \text{ és } \quad \partial_2f(0,0) = \left(\frac{d}{dy}f(0,y)\right)_{y=0} = \left(\frac{d}{dy}0\right)_{y=0} = 0$$
	Ezzel:
	$$\frac{f((0,0)+(h_1,h_2))-\overbrace{f(0,0)}^{0}-\begin{bmatrix}0 & 0\end{bmatrix} \begin{bmatrix}h_1 \\ h_2 \end{bmatrix}}{\norm{h}} = \frac{f(h_1,h_2)}{\norm{h}} - \frac{h_1h_2^2}{(h_1^2+h_2^2) = \norm{h}} = \frac{h_1h_2^2}{(\sqrt{h_1^2+h_2^2})^3} \overset{h_1=h_2}{=} $$
	$$\overset{h_1=h_2}{=} \frac{h_1h_1^2}{(\sqrt{h_1^2+h_1^2})^3} = \frac{h_1^3}{(\sqrt{2h_1^2})^3} = \frac{h_1^3}{\sqrt{8} \cdot |h_1|^3} \overset{h > 0}{=} \frac{h_1^3}{\sqrt{8} \cdot h_1^3} = \frac{1}{\sqrt{8}} \centernot{\xrightarrow[]{h_1 \to 0}} 0$$
	
	\textbf{HF./1.} A definíció alapján lássa be, hogy az $f(x,y)=x^3+xy \quad ((x,y) \in \R^2)$ függvény deriválható az $a := (2,3)$ pontban, és adja meg $f'(a)$-t. Az $f'(a)$-ra így kapott eredményt ellenőrizze a Jacobi-mátrix kiszámításával.
	
	\begin{framed}
		$$f \in D(a) \Longleftrightarrow \exists A \in \R^{m \times n} \quad \lim\limits_{h \to 0} \frac{f(a+h)-f(a)-A\cdot h}{\norm{h}} = 0$$
		$$\varepsilon(h) = f(a+h)-f(a)-A\cdot h \quad \longrightarrow \quad \varepsilon(h) + A\cdot h = f(a+h)-f(a)$$
	\end{framed}

	$f(a)=f(2,3)=2^3+6=14$
	
	$f(a+h)-f(a)=f(2+h_1,3+h_2)-14=(1+h_1)^3+(1+h_1)(3+h_2)-14=2^3+2 \cdot 2h_1 + 2h_1^2+h_1 \cdot 2^2 + 2 \cdot 2h_1^2+h_1^3+6+2h_2+3h_1+h_1h_2-14=15h_1+2h_2+6h_1^2+h_1^3+h_1h_2=\begin{bmatrix}15 & 2\end{bmatrix} \begin{bmatrix}h_1\\h_2\end{bmatrix}\underbrace{+6h_1^2+h_1^3+h_1h_2}_{\varepsilon(h)}$
	$$\left|\frac{\varepsilon(h)}{\norm{h}}-0\right|=\frac{|\varepsilon(h)|}{\norm{h}}=\frac{|6h_1^2+h_1^3+h_1h_2|}{\norm{h}} \le \frac{6\norm{h}^2+\norm{h}^3+3\norm{h} \cdot \norm{h}}{\norm{h}} = \frac{9\norm{h}^2+\norm{h}^3}{\norm{h}} \le $$
	$$\le \frac{10\norm{h}^3}{\norm{h}} = 10\norm{h}^2 = 10\norm{h-(0,0)}^2 \xrightarrow[]{h \to 0} 0\text{.}$$
	Tehát $f \in D(a)$ és $f'(a) = \begin{bmatrix}15 & 2\end{bmatrix} \in \R^{1 \times 2}$.
	
	\textit{Ell.:}
	$$\partial_1f = 3x^2+y \quad \rightarrow \quad \partial_1f(2,3)=3 \cdot 2^2 + 3 = 12 + 3 = 15$$
	$$\partial_2f = x \quad \rightarrow \quad \partial_1f(2,3)=2$$
	
	\newpage
	
	%\textbf{HF./2.}%
	
	%\textbf{a)} Mutassa meg, hogy $f \in C\{(0,0)\}$.%
	%$$\lim\limits_{(x,y) \to (0,0)} \frac{xy}{\sqrt{x^2+y^2}} = 0 \text{, mivel } %\left|\frac{xy}{\sqrt{x^2+y^2}} - 0\right| = \frac{|x||y|}{\norm{(x^2,y^2)}^{\frac{1}{2}}}$$
	
\end{document}