\documentclass[a4paper]{article}
\usepackage[textwidth=170mm, textheight=230mm, margin=20mm, bottom=25mm]{geometry}
\usepackage[normalem]{ulem}
\usepackage[utf8]{inputenc}
\usepackage[T1]{fontenc}
\usepackage[magyar]{babel}
\usepackage{amsmath, amsthm, amssymb, hyperref}
\usepackage[hu]{datetime}
\usepackage{enumitem}
\usepackage{lmodern}
\usepackage{xparse}
\usepackage{multicol}
\usepackage{framed}

\DeclareSymbolFont{lettersA}{U}{txmia}{m}{it}
\makeatletter
\DeclareMathSymbol{\m@thbbch@rC}{\mathord}{lettersA}{131}
\DeclareMathSymbol{\m@thbbch@rN}{\mathord}{lettersA}{142}
\DeclareMathSymbol{\m@thbbch@rQ}{\mathord}{lettersA}{145}
\DeclareMathSymbol{\m@thbbch@rR}{\mathord}{lettersA}{146}
\DeclareMathSymbol{\m@thbbch@rZ}{\mathord}{lettersA}{154}
\makeatother
\ExplSyntaxOn
\NewDocumentCommand{\varmathbb}{m}
{
	\tl_map_inline:nn { #1 } { \use:c { m@thbbch@r##1 } }
}
\ExplSyntaxOff

\input{glyphtounicode}
\pdfgentounicode=1
\hypersetup{colorlinks = true}
\setlength\parindent{0pt}
\setlength{\parskip}{1em}
\def\Z{\mathbb{Z}}
\def\Q{\mathbb{Q}}
\def\R{\mathbb{R}}
\def\C{\mathbb{C}}
\def\N{\mathbb{N}}

\def\ZZ{\varmathbb{Z}}
\def\QQ{\varmathbb{Q}}
\def\RR{\varmathbb{R}}
\def\CC{\varmathbb{C}}
\def\NN{\varmathbb{N}}

\newtheoremstyle{qstyle}{1.5em}{-1em}{\bfseries\boldmath}{}{\unboldmath\bfseries}{.}{ }{}
\theoremstyle{qstyle}
\newtheorem{question}{}{}
\setlist[itemize]{topsep=0pt}

\begin{document}
	\begin{center}
		{\Large\textbf{Analízis 3. (B és C szakirány)}}\\
		{\Large Szükséges ismeretek a 3. gyakorlathoz}
	\end{center}

	\begin{framed}
		\textbf{Jelen dokumentum ekkor lett frissítve: {\yyyymmdddate\today} \ \currenttime}\\
		További kidolgozások elérhetőek \href{https://people.inf.elte.hu/totadavid95/Analizis3/2019-tavasz/}{ide kattintva}. A gyakorlatok anyaga \href{http://numanal.inf.elte.hu/~szili/Oktatas/An3_BC_szakirany_2019/An3_BC_gyak_2019_tavasz.pdf}{ide kattintva} érhető el.\\	
		Forrás(ok): \href{http://numanal.inf.elte.hu/~szili/Oktatas/An3_BC_szakirany_2019/An3_BC_gyak_2019_tavasz.pdf}{Dr. Szili László - Analízis 3. gyakorlatok}, \href{https://people.inf.elte.hu/totadavid95/Analizis3/Anal3_def.pdf}{2018 őszi kidolgozás}
	\end{framed}

	\begin{question}
		Milyen \emph{elégséges} feltételt ismer primitív függvény létezésére?
	\end{question}
	Ha $I\subset\R$ nyílt intervallum és $f:I\to\R$ folytonos függvény, akkor $f$-nek létezik primitív függvénye.
	
	\begin{question}
		Adjon meg olyan függvényt, amelynek \emph{nincs} primitív függvénye.
	\end{question}
	$f(x)=\mathrm{sign}(x)$   \quad  $(x \in \left(-1,1)\right)$
	
	\begin{question}
		Milyen állítást ismer a primitív függvények számával kapcsolatban?
	\end{question}
	Tegyük fel, hogy $I \subset \R$ nyílt intervallum és $f : I \to \R$ adott függvény. Ekkor
	\vspace{-4mm}
	\begin{enumerate}
		\item Ha $F : I \to \R$ a $f$ függvény egy primitív függvénye, akkor minden $c\in\R$ esetén az $F + c$ függvény is primitív függvénye $f$-nek.
		\item Ha $F_1,F_2 : I \to \R$ primitív függvényei a $f$ függvénynek, akkor
		$$\exists c\in\R : F_1(x)=F_2(x)+c \quad (x \in I)\text{,}$$
		azaz a primitív függvények konstansban különböznek egymástól.
	\end{enumerate}
	\vspace{-4mm}
	
	\begin{question}
		Mit ért a határozatlan integrál linearitásán?
	\end{question}
	Legyen $I\subset\R$ egy nyílt intervallum. Ha az $f,g : I\to\R$ függvényeknek létezik primitív függvénye, akkor tetszőleges $\alpha,\beta\in\R$ mellett $(\alpha f+\beta g)$-nek is létezik primitív függvénye és
	$$\int (\alpha f+\beta g) = \alpha \int f + \beta \int g \text{.}$$
	
	\begin{question}
		Fogalmazza meg a primitív függvényekkel kapcsolatos \emph{második helyettesítési szabályt}.
	\end{question}
	Legyen $I,J \subset \R$ nyílt intervallum; $g : I \to J$ bijekció, $g \in D(I)$, $g'(x) \neq 0 (x \in I)$; $f : J \to \R$ és $x_0 \in J$. Ha az $(f \circ g) \cdot g' : I \to \R$ függvénynek van primitív függvénye, akkor $f$-nek is van primitív függvénye és
	$$\int\limits_{x_0} f = \left(\; \int\limits_{g^{-1}(x_0)} (f \circ g) \cdot g' \right) \circ g^{-1} \text{.} $$
	
	\begin{question}
		Milyen ekvivalens átfogalmazást ismer a Riemann-integrálhatóságra a
		Riemann-féle közelítő összegekkel?

	\end{question}
	Legyen $a,b \in \R$, $a<b$. Ekkor $f\in R[a,b]$ és $\int_{a}^{b} f = I \Longleftrightarrow \forall \varepsilon > 0$ számhoz $\exists \delta > 0: \forall \tau \in \mathcal{F}[a,b], ||\tau || := \max\limits_{k = 0,\dots,n-1} (x_{k+1} - x_k) < \delta$ esetén az
	$$|\sigma (f,\tau,\xi)-I| < \varepsilon$$
	egyenlőtlenség teljesül a $\xi$ közbülső helyek tetszőleges megválasztása mellett.

	\begin{question}
		Írja le az integrálfüggvénnyel kapcsolatban tanult tételt.
	\end{question}
	Legyen $f\in R[a,b],x_0\in [a,b], F(x) := \int_{x_0}^{x} f(t) dt \quad (x \in [a,b])$. Ekkor
	\vspace{-4mm}
	\begin{enumerate}
		\item a $F$ integrálfüggvény folytonos $[a,b]$-n;
		\item ha $d\in (a,b)$ és $f$ folytonos $d$-ben, akkor $F$ differenciálható $d$-ben és $F'(d)=f(d)$.
	\end{enumerate}
	\vspace{-4mm}
	
\end{document}